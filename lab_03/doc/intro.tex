\chapter*{Введение}
\addcontentsline{toc}{chapter}{Введение}

\textbf{Сортировка} --- процесс перегруппировки заданной последовательности объектов в определенном порядке. Это одна из главных процедур обработки структурированной информации \cite{sort0}.

Алгоритмы сортировки имеют большое значение, так как позволяют эффективнее проводить работу с последовательностью данных. Например, возьмем задачу поиска элемента в последовательности --- при работе с отсортированным набором данных время, которое нужно на нахождение элемента, пропорционально логарифму количества элементов. Последовательность, данные которой расположены в хаотичном порядке, занимает время, которое пропорционально количеству элементов, что куда больше логарифма.

Целью данной лабораторной работы является изучение алгоритмов сортировки. Для достижения поставленной цели требуется выполнить следующие задачи:

\begin{enumerate}[label={\arabic*)}]
	\item Описание трех алгоритмов сортировки:
		\begin{itemize}
			\item Битонная сортировка;
			\item Поразрядная сортировка;
			\item Сортировка пузырьком.
		\end{itemize}
	\item Построение схемы рассматриваемых алгоритмов;
	\item Создание программного обеспечения, реализующего перечисленные алгоритмы;
	\item Проведение сравнительного анализа реализаций алгоритмов по затраченному процессорному времени и памяти;
	\item Описание и обоснование полученных результатов в отчете о выполненной лабораторной работе.
\end{enumerate}