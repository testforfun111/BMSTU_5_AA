\chapter*{Заключение}
\addcontentsline{toc}{chapter}{Заключение}

В результате исследования было получено, что при размере массива больше 128, алгоритм сортировки пузырьком работает медленее алгоритма битонной сортировки в 1.8 раза и алгоритма поразрядной сортировки в 10.8 раза.

Проведя анализ оценки затрат реализаций алгоритмов по памяти, можно сказать, что поразрядная сортировка больше затратны, так как для них использовать дополнительные массивы.
Затраты по памяти для алгоритма битонной сортировки и сортировки пузырьком не отличаются друг от друга.

Решены все поставленные задачи:
\begin{enumerate}[label={\arabic*)}]
	\item Описаны трех алгоритмов сортировки:
	\begin{itemize}
		\item Битонная сортировка;
		\item Поразрядная сортировка;
		\item Сортировка пузырьком.
	\end{itemize}
	\item Построены схемы рассматриваемых алгоритмов;
	\item Создано программного обеспечения, реализующего перечисленные алгоритмы;
	\item Проведен сравнительного анализа реализаций алгоритмов по затраченному процессорному времени и памяти.
	\item Подготовлен отчете о выполненной лабораторной работе.
\end{enumerate}

