\chapter{Аналитическая часть}

В данном разделе будут рассмотрены три алгоритма сортировок: битонная сортировка, поразрядная сортировка, сортировка пузырьком.

\section{Битонная сортировка}

\textbf{Битонная сортировка} (англ. Bitonic sorter) --- параллельный алгоритм сортировки данных, метод для создания сортировочной сети \cite{sort1}.
Разработан американским информатиком Кеннетом Бэтчером в 1968 году.
В основе алгоритма лежит понятие «битонной последовательности».
Название было выбрано по аналогии с монотонной последовательностью.

Алгоритм основан на сортировке битонных последовательностей.
Такой последовательностью называется последовательность, которая сначала монотонно не убывает, а затем монотонно не возрастает, либо приводится к такому виду в результате циклического cдвига.

Процесс битонного слияния преобразует битонную последовательность в полностью отсортированную последовательность. Алгоритм битонной сортировки состоит из применения битонных преобразований до тех пор, пока множество не будет полностью отсортировано.

\section{Поразрядная сортировка}

Люди изобрели много сортировок для разных данных и под разные задачи.
\textbf{Поразрядная сортировка} — это почти как сортировка по алфавиту, но для данных. В английском языке это называется Radix sort — сортировка по основанию системы счисления \cite{sort2}.

Алгоритм поразрядной сортировки гениален в том, что сортирует не числа целиком, а значения разрядов.
Получается, что он как бы разбирается с числами на уровне единиц, десятков, сотен и т. д. и только потом он делает общую сортировку.
Это позволяет ему не бегать по всем сравниваемым числам и не делать миллион сравнений.
Отсюда и экономия времени.

\section{Сортировка пузырьком}

\textbf{Сортировка пузырьком} — один из самых известных алгоритмов сортировки.
Здесь нужно последовательно сравнивать значения соседних элементов и менять числа местами, если предыдущее оказывается больше последующего \cite{sort3}.
Таким образом элементы с большими значениями оказываются в конце списка, а с меньшими остаются в начале.

\section{Вывод}
В данной работе стоит задача реализации 3 алгоритмов сортировки, а именно: битонная сортировка, поразрядная сортировка и сортировка пузырьком. 