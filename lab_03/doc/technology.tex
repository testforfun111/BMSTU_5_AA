\chapter{Технологическая часть}

В данном разделе будут указаны требования к программному обеспечению, средства реализации, будут представлены реализация алгоритмов, а также функциональные тесты.

\section{Требования к программному обеспечению}
К программе предъявляются следующие требования.
\begin{itemize}
	\item Программа должна предоставлять 2 режима работы: режим сортировки массива между введёнными пользователем и режим замера процессорного времени выполнения сортировки массива.
	\item В начале работы программы пользователю нужно ввести целое число --- это выбор пункта меню.
\end{itemize}

К первому режиму работы программы предъявляются следующие требования:
\begin{itemize}
	\item если пункт меню --- число от 1 до 3, то сортировать массив, для этого надо ввести размер массива и его элементы;
	\item программа должна вывести результирующий массив.
\end{itemize}

Ко второму режиму работы программы предъявляются следующие требования:
\begin{itemize}
	\item если пункт меню --- число от 4 до 5, то провести замеры времени и памяти сортировки каждого алгоритма;
	\item массив генерируются автоматически.
\end{itemize}

\section{Средства реализации}

Реализация данной лабораторной работы выполнялась при помощи языка программирования С. Данный выбор обусловлен наличием у языка функции $clock()$ измерения процессорного времени.

Визуализация графиков с помощью библиотеки $Matplolib$.
\section{Сведения о модулях программы}

Программа состоит из следующих модулей:

\begin{itemize}
	\item main.c --- главный файл программы, предоставляющий пользователю меню для выполнения основных функций;
	\item sort.c --- файл, содержащий реализации этих функций;
	\item sort.h --- заголовочный файл, содержащий объявлении функций, реализуюших рассматриваемых алгоритмови;
	\item time.c --- файл, содержащий функции замеров времени работы указанных алгоритмов;
	\item time.h --- заголовочный файл, содержащий объявлении функций замеров времени работы указанных алгоритмов;
	\item graph\_result.py --- файл, содержащий функции визуализации временных характеристик описанных алгоритмов;
	\item memory.c --- файл, содержащий функции замеров памяти для реализации алгоритмов;
	\item memory.h --- заголовочный файл, содержащий объявлении функций замеров памяти для реализации.
\end{itemize}

\section{Реализация алгоритмов}

Aлгоритм битонной сортировки, алгоритм поразрядной сортировки и алгоритм сортировки пузырьком приведены в листингах \ref{lst:bitonic} -- \ref{lst:bubble}.


\begin{center}
\captionsetup{justification=raggedright,singlelinecheck=off}
\begin{lstlisting}[label=lst:bitonic,caption=Aлгоритм битонной сортировки]
void compAndSwap(int *arr, int i, int j, int dir)
{
	if (dir==(arr[i]>arr[j]))
		swap(arr + i, arr + j);
}

void bitonicMerge(int *arr, int low, int cnt, int dir)
{
	if (cnt>1)
	{
		int k = cnt/2;
		for (int i=low; i<low+k; i++)
			compAndSwap(arr, i, i+k, dir);
		bitonicMerge(arr, low, k, dir);
		bitonicMerge(arr, low+k, k, dir);
	}
}

void bitonicSort(int *arr,int low, int cnt, int dir)
{
	if (cnt>1)
	{
		int k = cnt/2;
		
		bitonicSort(arr, low, k, 1);
		
		bitonicSort(arr, low+k, k, 0);
		
		bitonicMerge(arr, low, cnt, dir);
	}
}

void bitonic_sort(int *arr, int size)
{
	bitonicSort(arr, 0, size, 1);
}
\end{lstlisting}
\end{center}
\clearpage

\begin{center}
\captionsetup{justification=raggedright,singlelinecheck=off}
\begin{lstlisting}[label=lst:radix,caption=Aлгоритм поразрядной сортировки]
int getMax(int *arr, int n)
{
	int mx = *arr;
	for (int i = 1; i < n; i++)
		if (arr[i] > mx)
			mx = arr[i];
	return mx;
}

void countSort(int *arr, int n, int exp)
{
	int *output = allocate_arr(n);
	int i, count[10] = { 0 };
	
	for (i = 0; i < n; i++)
		count[(arr[i] / exp) % 10]++;
	
	for (i = 1; i < 10; i++)
		count[i] += count[i - 1];
	
	for (i = n - 1; i >= 0; i--) {
		output[count[(arr[i] / exp) % 10] - 1] = arr[i];
		count[(arr[i] / exp) % 10]--;
	}
	
	for (i = 0; i < n; i++)
		arr[i] = output[i];
	deallocate_arr(output);
}

void radix_sort(int *arr, int n)
{
	int m = getMax(arr, n);
	
	for (int exp = 1; m / exp > 0; exp *= 10)
		countSort(arr, n, exp);
}
\end{lstlisting}
\end{center}
\clearpage
\begin{center}
	\captionsetup{justification=raggedright,singlelinecheck=off}
	\begin{lstlisting}[label=lst:bubble,caption=Aлгоритма сортировки пузырьком]
		void bubble_sort(int *arr, int size)
		{
			for (int i = 0; i < size; i++)
				for (int j = 0; j < size - i - 1; j++)
					if (arr[j] > arr[j + 1])
						swap(arr + j, arr + j + 1);
		}
	\end{lstlisting} 
\end{center}


\section{Функциональные тесты}

В таблице \ref{tbl:tests} приведены функциональные тесты для функций, реализующих алгоритмы сортировки. Все тесты пройдены успешно.

\begin{table}[h]
	\begin{center}
		\begin{threeparttable}
			\captionsetup{justification=raggedright,singlelinecheck=off}
			\caption{\label{tbl:tests} Функциональные тесты}
			\begin{tabular}{|c|c|c|c|c|}
				\hline
				&Входные данные& \multicolumn{3}{c|}{Ожидаемый результат} \\
				\hline
				№&Массив&Битонная&Поразрядрая&Пузырьком \\
				\hline
				1&[ ]&[ ]&[ ]&[ ] \\
				\hline
				2&[1]&[1]&[1]&[1] \\
				\hline
				3&[3, 3, 3, 3]&[3, 3, 3, 3]&[3, 3, 3, 3]&[3, 3, 3, 3] \\
				\hline
				4&[1, 2, 3, 4]&[1, 2, 3, 4]&[1, 2, 3, 4]&[1, 2, 3, 4] \\
				\hline
				5&[9, 5, 3, 1]&[1, 3, 5, 9]&[1, 3, 5, 9]&[1, 3, 5, 9] \\
				\hline
				6&[9, 5, 6, 1]&[1, 5, 6, 9]&[1, 5, 6, 9]&[1, 5, 6, 9] \\
				\hline
			\end{tabular}
		\end{threeparttable}
	\end{center}
\end{table}


\section{Вывод}

Были реализованы функции всех алгоритмов сортировок — битонной, поразрядной и пузырьком.
Было проведено функциональное тестирование указанных функций.