\chapter{Конструкторская часть}
В данном разделе будут представлены требования к программе, схемы алгоритмов поиска расстояний Левенштейна и Дамерау~---~Левенштейна, выбранные типы данных.


\section{Требования к программе}
К программе предъявляются следующие требования.
\begin{itemize}
\item Программа должна предоставлять 2 режима работы: режим расчёта расстояний между введёнными пользователем двумя строками и режим массированного замера процессорного времени выполнения поиска редакционных расстояний реализациями алгоритмов.
\item В начале работы программы пользователю нужно ввести целое число --- это выбор пункта меню.
\end{itemize}

К первому режиму работы программы предъявляются следующие требования:
\begin{itemize}
	\item если пункт меню --- число от 1 до 4, то вычислить расстояние, для этого надо ввести две строки;
	\item две пустые строки --- корректный ввод, программа не должна аварийно завершаться;
	\item программа должна вывести расстояние и, если расстояние вычисляется не рекурсивно, также матрицу D.
\end{itemize}

Ко второму режиму работы программы предъявляются следующие требования:
\begin{itemize}
	\item если пункт меню --- число 5, то провести замеры времени поиска каждого расстояния реализацией каждого алгоритма;
	\item длины строк во втором режиме принимаются равными;
	\item строки заданной длины генерируются автоматически.
\end{itemize}


\section{Алгоритмы поиска редакционных расстояний}
Схемы алгоритмов поиска редакционных расстояний представлены на рисунках  \ref{img:lev}--\ref{img:recwithcache}. Перед началом расчёта по рекурсивному алгоритму поиска расстояния Дамерау~---~Левенштейна с кешем требуется инициализировать ячейки матрицы значениями $-1$.
\imgScale{0.5}{lev}{Схема итерационного алгоритма поиска расстояния Левенштейна}
\imgScale{0.5}{DamLev}{Схема итерационного алгоритма поиска расстояния Дамерау~---~Левенштейна}
\imgScale{0.45}{recDamLev}{Схема рекурсивного алгоритма поиска расстояния  Дамерау~---~Левенштейна}
\imgScale{0.5}{recwithcache}{Схема рекурсивного алгоритма поиска расстояния Дамерау~---~Левенштейна с кешем}
\clearpage

\section*{Вывод}
Перечислены требования программе, а также на основе теортических данных, полученных из аналитического раздела, были построены схемы требуемых алгоритмов.
