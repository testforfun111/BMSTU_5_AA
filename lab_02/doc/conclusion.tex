\chapter*{Заключение}
\addcontentsline{toc}{chapter}{Заключение}

В результате исследования было получено, что при размере матриц, большем 50, небходимо использовать оптимизированный алгоритм умножения матриц по Винограду, так как данный алгоритм работает быстрее стандартного алгоритма в 1.1 раза. При этом стандартный алгоритм быстрее алгоритма Винограда в 1.2 раза.

Кроме того алгоритм Винограда предпочтительно использовать для умножения матриц четных размеров, так как указанный алгоритм работает в 1.2 раза быстрее, чем на матрицах с нечетным размером. Это связано с проведением дополнительных вычислений для крайних строк и столбцов.

Цель, поставленная перед началом работы, была достигнута. В ходе лабораторной работы были решены следующие задачи:

\begin{itemize}
	\item были изучены классический алгоритм, алгоритм Винограда, его оптимизированная версия умножения матриц и алгоритм Штрассена;
	\item были разработаны изученные алгоритмы;
	\item был проведен сравнительный анализ реализованных алгоритмов;
	\item был подготовлен отчет о выполненной лабораторной работе.
\end{itemize}
