\chapter{Аналитическая часть}

В данном разделе будут представлены описания последовательного и параллельного вариантов алгоритма поиска подстроки в строке полным перебором.

\section{Последовательный алгоритм поиска подстроки в строке}
Данный алгоритм можно определить следующим образом \cite{alg}. 
Пусть задана строка S из N элементов и строка p из M элементов. 

Описаны они так: $string S[N], P[M];$


Задача поиска подстроки P в строке S заключается в нахождении первого слева вхождения P в S, т.е. найти значение индекса i, начиная с которого 

$S[i] = P[0], S[i + 1] = P[1],…, S[i + M - 1] = P[M - 1]$

\section{Параллельный алгоритм поиска подстроки в строке}
Идея алгоритма состоит в том, чтобы разделить строку на подстроку, соответствующие каждому потоку, а затем выполнить поиска для каждой подстроку.
Найденные индексы сохраняются в массиве, поэтому может возникнуть ошибка, поскольку к массиву может обращаться несколько потоки. 
Для этого использовать мьютекс  для организации монопольного доступа к массиву.

Основная проблема этой версии заключается в том, что если паттерн поступает в часть разделения данных или точку соединения, он не обнаруживается, поскольку данные обрабатываются на разных потоках.
Для решения этой проблемы мы обрабатываем еще часть строки в каждой точке соединения.

\section*{Вывод}

Были изучены последовательный и параллельный алгоритмы поиска подстроки в строке.