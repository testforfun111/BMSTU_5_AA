\chapter*{\centering Введение}
\addcontentsline{toc}{chapter}{Введение}

В настоящее время компьютерные системы оперируют большими объемами данных. 
Над этими данными проводится большой объем различного рода вычислений. 
Для того, чтобы они выполнянлись быстрее, было придумано параллельное программирование.
Параллельное программирование --- это подход к разработке программ, в котором задачи выполняются на нескольких процессорах или ядрах процессора \cite{multi}.

Его суть заключается в том, чтобы относительно равномерно разделять нагрузку между потоками ядра.
Каждое из ядер процессора может обрабатывать по одному потоку, поэтому когда количество потоков на ядро становится больше, происходит квантование времени.
Это означает, что на каждый процесс выделяется фиксированная величина времени (квант), после чего в течение кванта обрабатывается следующий процесс. 
Таким образом создается видимость параллельности. 
Тем не менее, данная оптимизация может сильно ускорить вычисления \cite{multi2}.

Целью данной лабораторной работы является получение навыков параллельного программирования на базе алгоритма поиска подстроки в строке полным перебором.
Для достижения поставленной цели требуется выполнить следующие задачи:

\begin{itemize}
	\item описать последовательный и параллельный алгоритмы поиска подстроки в строке полным перебором;
	\item построить схемы данных алгоритмов;
	\item создать программное обеспечение, реализующее рассматриваемые алгоритмы;
	\item провести сравнительный анализ по времени для реализованного алгоритма;
	\item подготовить отчет о выполненной лабораторной работе.
\end{itemize}