\chapter{Исследовательская часть}

В данном разделе будут приведены демонстрации работы программы, и будет проведен сравнительный анализ реализованного алгоритма поиска подстроки в строке полным перебором по времени.

\section{Технические характеристики}

Ниже представлены характеристики компьютера, на котором проводились замеры времени  работы реализации алгоритмов:

\begin{itemize}
	\item операционная система Windows 10 Домашняя 21H2;
	\item оперативная память 12 Гб;
	\item процессор Intel(R) Core(TM) i7-9750H CPU @ 2.6ГГц.
\end{itemize}


\section{Демонстрация работы программы}

На рисунках \ref{img:demon1} и \ref{img:demon2} представлен результат работы программы. В каждом примере пользователем введены строка и подстрока и получены все индексы вхождения.

\img{50mm}{demon1}{Демонстрация работы программы при последовательном алгоритме}
\img{50mm}{demon2}{Демонстрация работы программы при параллельном алгоритме}

\section{Время выполнения реализаций алгоритмов}
Результаты замеров времени работы реализаций алгоритмов поиска подстроки в строке преведены в таблице \ref{tbl:4patok}.
Для параллельной реализации алгоритма выбрано количество потоков, равное 12.
\begin{table}[ht]
	\small
	\begin{center}
		\begin{threeparttable}
			\caption{Результаты замеров времени}
			\label{tbl:4patok}
			\begin{tabular}{|c|c|c|c|}
				\hline
				\bfseries Размер строки & \bfseries Размер подстроки & \bfseries Без многопоточности & \bfseries c 12 потоками
				\csvreader{csv/4patok.csv}{} 
				{\\\hline \csvcoli & \csvcolii & \csvcoliii & \csvcoliv} \\
				\hline
			\end{tabular}	
		\end{threeparttable}
	\end{center}
\end{table}

По таблице~\ref{tbl:4patok} был построен график, который иллюстрирует зависимость времени, затраченного реализациями алгоритмов поиска подстроки в строке полным перебором, от размера строки --- рис.~\ref{img:12threads}.

\img{100mm}{12threads}{Сравнение времени работы алгоритма без распараллеливания и с 12 вспомогательными потоками при разных размерах строки}
\clearpage

В таблице~\ref{tbl:manypatok} приведены результаты замеров по времени параллельного алгоритма поиска подстроки в строке полным перебором при разном количество потоков. 

\begin{table}[ht]
	\small
	\begin{center}
		\begin{threeparttable}
			\caption{Результаты замеров времени}
			\label{tbl:manypatok}
			\begin{tabular}{|c|c|}
				\hline
				\bfseries Количество потоков & \bfseries Время выполнения
				\csvreader{csv/manypatok.csv}{} 
				{\\\hline \csvcoli & \csvcolii}\\
				\hline
			\end{tabular}	
		\end{threeparttable}
	\end{center}
\end{table}

По таблице~\ref{tbl:manypatok} был построен график, который иллюстрирует зависимость времени, затраченного реализацией параллельного алгоритма поиска подстроки в строке полным перебором, от количества потоков --- рис.~\ref{img:manythreads}.

\img{100mm}{manythreads}{Сравнение времени работы алгоритма с распараллеливанием на различное количество потоков при размере строки 5000000}

Исходя из этих данных можно понять, параллельный алгоритм работает быстрее всего с 12 потоками.

\section{Вывод}
По графикам видно, что параллельный алгоритм с использованием 12 потоков, работает быстрее, чем последовательный алгоритм поиска подстроки в строке полным перебором.
Сравнивая параллельный алгоритм с разным количеством потоков видно, что количество потоков 12 обеспечивает наилучшую производительность.