\chapter{Технологическая часть}

В данном разделе будут указаны требования к программному обеспечению, средства реализации, будут представлены реализации алгоритмов, а также функциональные тесты.

\section{Требования к программному обеспечению}

К программе предъявляются следующие требования.
\begin{itemize}
	\item Программа должна предоставлять 2 режима работы: режим поиска подстроки в строке и режим замера процессорного времени выполнения поиска подстроки в строке.
	\item В начале работы программы пользователю нужно ввести целое число --- это выбор пункта меню.
\end{itemize}

К первому режиму работы программы предъявляются следующие требования:
\begin{itemize}
	\item если пункт меню --- число от 1 до 2, то искать подстроки в строке, для этого надо ввести две строки;
	\item программа должна вывести индексы в заданной строке.
\end{itemize}

Ко второму режиму работы программы предъявляются следующие требования:
\begin{itemize}
	\item если пункт меню --- число 3, то провести замеры времени поиска подстроки в строке каждого алгоритма;
	\item строки генерируются автоматически.
\end{itemize}


\section{Средства реализации}
Реализация данной лабораторной работы выполнялась при помощи языка программирования С++. Данный выбор обусловлен наличием у языка функции $clock()$ измерения процессорного времени.

Визуализация графиков с помощью библиотеки $Matplotlib$ \cite {matplot}.

\section{Сведения о модулях программы}

Программа состоит из следующих модулей:

\begin{itemize}
	\item main.cpp --- точка входа программы;
	\item algorithm.h --- заголовочный файл, содержащий объявлении функций, реализуюших рассматриваемых алгоритмов;
	\item algorithm.cpp --- файл, содержащий реализации этих функций;
	\item measure.h --- заголовочный файл, содержащий объявлении функций замеров времени работы рассматриваемых алгоритмов;
	\item measure.cpp ---algo файл, содержащий реализации функций замеров времени работы рассматриваемых алгоритмов.
\end{itemize}


\section{Реализация алгоритмов}
Реализации последовательного и параллельного алгоритмов приведены в листингах \ref{lst:algoS}--\ref{lst:thread}.

\begin{lstinputlisting}[
	caption={Последовательный алгоритм поиска подстроки в строке полным перебором},
	label={lst:algoS},
	linerange={4-28}
	]{../src/algorithm.cpp}
\end{lstinputlisting}

\begin{lstinputlisting}[
	caption={Параллельный алгоритм поиска подстроки в строке полным перебором},
	label={lst:radix},
	linerange={60-87}
	]{../src/algorithm.cpp}
\end{lstinputlisting}

\begin{lstinputlisting}[
	caption={Задача одного потока},
	label={lst:thread},
	linerange={31-57}
	]{../src/algorithm.cpp}
\end{lstinputlisting}




\section{Функциональные тесты}

В таблице \ref{tbl:tests} приведены функциональные тесты для функции, реализующей алгоритм поиска подстроки в строке. Все тесты пройдены успешно.

\begin{table}[h]
	\begin{center}
		\begin{threeparttable}
			\captionsetup{justification=raggedright,singlelinecheck=off}
			\caption{\label{tbl:tests} Функциональные тесты}
			\begin{tabular}{|c|c|c|c|}
				\hline
				& \multicolumn{2}{c|}{\bfseries Входные данные}& \bfseries Ожидаемый результат \\
				\hline
				№&\bfseries Строка& \bfseries Подстрока& \makecell{\bfseries Алгоритм\\ \bfseries полным перебором} \\
				\hline
				1&&&Сообщение об ошибке\\
				\hline
				2&anhyeuem&em&6\\
				\hline
				3&abcdabc&bc& 1 5 \\
				\hline
				4&anhyeuem&z&\\
				\hline
			\end{tabular}
		\end{threeparttable}
	\end{center}
\end{table}

\section*{Вывод}

Были представлены реализаций двух версий алгоритма поиска подстроки в строке полным перебором --- последовательного и параллельного. Также в данном разделе была приведены функциональные тесты.
