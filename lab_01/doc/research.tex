\chapter{Исследовательская часть}

В данном разделе будут представлены примеры работы программы, проведены замеры процессорного времени и предоставлена информация о технических характеристиках устройства.

\section{Технические характеристики устройства}

Ниже представлены характеристики компьютера, на котором проводились замеры времени  работы реализации алгоритмов:

\begin{itemize}
    \item операционная система Windows 10 Домашняя 21H2;
    \item оперативная память 12 Гб;
    \item процессор Intel(R) Core(TM) i7-9750H CPU @ 2.6Гц.
\end{itemize}

Загруженность компонентов:

\begin{itemize}
    \item процессор - 27\%;
    \item оперативная память - 53\%
\end{itemize}


\section{Примеры работы программы}

На рисунках \ref{img:exLev}-\ref{img:exCurDDL} представлен результат работы программы. В каждом примере пользователем введены две строки и получены результаты вычислений редакционных расстояний.

\imgScale{0.6}{exLev}{Пример работы программы (1)}
\imgScale{0.7}{exCurDDL}{Пример работы программы (2)}
\clearpage


\section{Время выполнения реализации алгоритмов}

Для замера процессорного времени выполнения реализации алгоритмов использовалась функция \textit{process\_time()} библиотеки \textit{time}. Возвращаемый результат - время в миллисекундах, число типа \textit{float}.
Чтобы получить достаточно точное значение, производилось усреднение времени.
Количество запусков замера процессорного времени 10000 раз.

В замерах использовались строки длиной от 0 до 7 символов, длины строк совпадают, строки заполняются случайными символами. Количество повторных измерений времени варьировалось от 50 до 1000. в зависимости от длин входных строк.

\begin{table}[ht]
	\small
	\begin{center}
		\begin{threeparttable}
			\caption{Результаты замеров процессорного времени}
			\label{tbl:time_random}
			\begin{tabular}{|c|c|c|c|c|}
				\hline
				& \multicolumn{4}{c|}{\bfseries Выходные данные} \\\cline{2-5}
				&&\multicolumn{3}{c|}{\bfseries Дамерау — Левенштейн}\\\cline{3-5}
				\bfseries Длина & \bfseries Левенштейн & \bfseries Нерекурсивный & \multicolumn{2}{c|}{\bfseries Рекурсивный}\\\cline{4-5}
				&\bfseries&\bfseries&\bfseries Без кэша&\bfseries С кэшем
				\csvreader{sort_time.csv}{}
				{\\\hline\csvcoli & \csvcolii & \csvcoliii & \csvcoliv & \csvcolv } \\
				\hline
			\end{tabular}
		\end{threeparttable}
	\end{center}
\end{table}

На рисунках \ref{img:gLevDDL}--\ref{img:gDDL2} также приведены результаты замеров процессорного времени.
\imgScale{0.6}{gLevDDL}{Зависимость времени выполнения реализации итерационных алгоритмов поиска расстояний Левенштейна от длины строки}
\FloatBarrier
\imgScale{0.6}{gDDL}{Зависимость времени выполнения реализации алгоритмов поиска расстояний Дамерау~---~Левенштейна от длины строки}
\FloatBarrier
\imgScale{0.6}{gDDL2}{Зависимость времени выполнения реализации итерационного алгоритма поиска расстояний Дамерау~---~Левенштейна и рекурсивного с кешем от длины строки}
\FloatBarrier

Вывод: Рекурсивный алгоритм поиска расстояния Дамерау~---~Левенштейна работает дольше всех; итерационный алгоритмов поиска расстояний Левенштейна работает бытсрее всехю
 
\section{Использование памяти}

Затраты по памяти для реализвции алгоритмов поиска расстояния Левенштейна (итерационного) и Дамерау~---~Левенштейна (итерационного) не отличаются друг от друга:
\begin{itemize}
	\item матрица: $(n + 1) \cdot (m + 1) \cdot \text{sizeof}(\text{int})$;
	\item строки str\_1, str\_2: $(n + m + 2) \cdot \text{sizeof}(\text{char})$;
	\item длины строк n, m: $2\cdot \text{sizeof}(\text{int})$;
	\item дополнительные переменные (i, j): $2\cdot \text{sizeof}(\text{int})$;
	\item адрес возврата.
\end{itemize}

Итого:
\begin{equation}
	\label{eq:memLev}
	(n + 1)(m + 1) \cdot \text{sizeof}(\text{int}) + (n + m + 2) \cdot \text{sizeof}(\text{char}) + 4 \cdot \text{sizeof}(\text{int})
\end{equation}

Затраты по памяти для алгоритма поиска расстояния Дамерау~---~Левенштейна (рекурсивный), для одного вызова:
\begin{itemize}
	\item строки str\_1, str\_2: $(n + m + 2) \cdot \text{sizeof}(\text{char})$;
	\item длины строк n, m: $2 \cdot \text{sizeof}(\text{int})$;
	\item дополнительные переменные (add, delete, change, result): $4 \cdot \text{sizeof}(\text{int})$;
	\item адрес возврата.
\end{itemize}

K - максимальное количество вызовов рекурсии:

\begin{equation}
 K = M + N
\end{equation}

Итого:
\begin{equation}
	\label{eq:memrecLev}
	((n + m + 2) \cdot \text{sizeof}(\text{char}) + 6 \cdot \text{sizeof}(\text{int})) \cdot (M + N)
\end{equation}


Затраты по памяти для алгоритма поиска расстояния Дамерау~---~Левенштейна (рекурсивный с кешем), для одного вызова:
\begin{itemize}
	\item строки str\_1, str\_2: $(n + m + 2) \cdot \text{sizeof}(\text{char})$;
	\item длины строк n, m: $2 \cdot \text{sizeof}(\text{int})$;
	\item дополнительные переменные (add, delete, change, result): $4 \cdot \text{sizeof}(\text{int})t$;
	\item адрес возврата;
\end{itemize}
   и матрица - $(n + 1) \cdot (m + 1) \cdot \text{sizeof}(\text{int})$.\newline

K - максимальное количество вызовов рекурсии:

\begin{equation}
	K = M + N
\end{equation}

Итого:
\begin{equation}
	\label{eq:memDLcache}
	((n + m + 2) \cdot \text{sizeof}(\text{char}) + 6 \cdot \text{sizeof}(\text{int})) \cdot (M + N) + (n + 1) \cdot (m + 1) \cdot \text{sizeof}(\text{int})
\end{equation}


На рисунках \ref{img:mem} также приведены результаты замеров памяти. 

\clearpage
\imgScale{0.6}{mem}{Замеры памяти, которую затрачивают реализации алогритмов}

\section{Вывод}

В результате замеров процессорного времени выделены следущие аспекты:
\begin{itemize}
    \item в результате эксперимента было получено, что при длине строк в более 4
    символов, алгоритм Левенштейна быстрее Дамерау~---~Левенштейна в 16\%;
    \item рекурсивный алгоритм поиска расстояния Дамерау~---~Левенштейна уже при длине строк, равной 4, символа проигрывает в 15 раз по времени итерационной и в 10 раз рекурсивной с кешем реализациям;
    \item итерационный алгоритм поиска расстояний Дамерау~---~Левенштейна в среднем на 49~\%~--~54~\% быстрее рекурсивного с кешем для длин строк от 0 до 7 символов;
\end{itemize}

Проведя анализ оценки затрат реализаций алгоритмов по памяти, можно сказать, что рекурсивные алгоритмы менее затратны, так как для них максимальный размер памяти растет прямо пропорционально сумме длин строк.

