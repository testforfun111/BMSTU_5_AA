\chapter{Аналитическая часть}

\section{Расстояние Левенштейна}
\textbf{Расстояние Левенштейна} --- минимальное количество редакционных операций, необходимых для преобразования одной строки в другую.\newline

Пусть $S_{1}$ и $S_{2}$ --- две строки, длиной $N$ и $M$ соответственно, а редакционные операции: %$M \cdot N$
\begin{itemize}
        \item вставка символа в произвольной позиции (I --- Insert);
        \item удаление символа в произвольной позиции (D --- Delete);
        \item замена символа на другой (R --- Replace). \newline
\end{itemize} 

Принято, что для этих операций ``штраф'' равен 1. \newline

Для поиска расстояния Левенштейна используют рекуррентную формулу, то есть такую, которая использует предыдущие члены ряда или последовательности для вычисления последующих:


\begin{equation}
	\label{eq:Lev}
	D(i, j) = \begin{cases}
		
		0, &\text{i = 0, j = 0}\\
		i, &\text{j = 0, i > 0}\\
		j, &\text{i = 0, j > 0}\\
		\min \lbrace \\
		\qquad D(i, j-1) + 1,\\
		\qquad D(i-1, j) + 1, &\text{i > 0, j > 0}\\
		\qquad D(i-1, j-1) + \text{m(a[i], b[j])}\\
		\rbrace,
	\end{cases}
\end{equation}


\begin{equation}
	\label{eq:label}
	m(a[i], b[j]) = \begin{cases}
		0, &\text{если $S_{i}$ = $S_{j}$,}\\
        1, &\text{иначе}
    \end{cases}
\end{equation}

Итерационный алгоритм поиска расстояния Левенштейна будет выполнять расчёт по формуле (\ref{eq:Lev}).

\section{Расстояние Дамерау~--~Левенштейна}
Расстояние Дамерау~---~Левенштейна измеряет минимальное количество операций (вставок, удалений, замен и перестановок символов), необходимых для преобразования одной строки в другую. Эта метрика часто используется в информационной теории, лингвистике и компьютерной науке для сравнения и анализа текстовых данных.

По сравнению с расстоянием Левенштейна, расстояние Дамерау~---~Левенштейна учитывает возможность перестановки символов, что может быть полезным, например, при исправлении опечаток, где буквы могут быть перепутаны местами в словах. Это позволяет более точно измерить сходство между двумя строками и более эффективно обрабатывать различные виды текстовых данных.

Тогда для данного расстояния определены следующие редакционные операции:
\begin{itemize}
        \item вставка символа в произвольной позиции (I - Insert);
        \item удаление символа в произвольной позиции (D - Delete);
        \item замена символа на другой (R - Replace);
        \item транспозиция двух символов (S - Swap).
\end{itemize} 

Таким образом, чтобы получить формулу нахождения расстояния Дамерау~---~Левенштейна, необходимо в формулу для поиска расстояния Левенштейна добавить еще одно определение минимума, но уже из четырех вариантов, если возможна перестановка двух соседних символов:

\begin{equation}
	\label{eq:D}
	D(i, j) = \begin{cases}
		max(i, j), &если min(i, j) = 0\\

		\min \lbrace \\
		\qquad D(i, j-1) + 1,&\text{i > 1, j > 1,}\\
		\qquad D(i-1, j) + 1,&\text{$S_{i}$ = $S_{j-1}$,}\\
		\qquad D(i-2, j-2) + 1,&\text{$S_{i-1}$ = $S_{j}$,}\\
		\qquad D(i-1, j-1) + \begin{cases}
                        		0, &\text{если $S_{i}$ = $S_{j}$,}\\
                        		1, &\text{иначе}
                        	\end{cases}\\
		\rbrace,\\
		
		\min \lbrace \\
		\qquad D(i, j-1) + 1,\\
		\qquad D(i-1, j) + 1,&\text{иначе}\\
		\qquad D(i-1, j-1) + \begin{cases}
                        		0, &\text{если $S_{i}$ = $S_{j}$,}\\
                        		1, &\text{иначе}
                        	\end{cases}\\
		\rbrace
	\end{cases}
\end{equation}

\subsection{Итерационный алгоритм поиска расстояния Дамерау~---~Левенштейна}
Суть итерационного алгоритма поиска расстояния Дамерау~---~Левен\-штейна заключается в построчном заполнении матрицы промежуточных значений расстояния. Результатом является последний элемент последней строки матрицы. При больших строках приходится хранить большие матрицы. Поскольку нам не важны промежуточные значения, можно хранить только текущую и предыдущую строки. Итерационный алгоритм поиска расстояния Дамерау~---~Левенштейна будет выполнять расчёт по формуле~(\ref{eq:D}).

\subsection{Рекурсивный алгоритм поиска расстояния \\ \mbox{Дамерау~---~Левенштейна}}
Данный алгоритм использует для решения формулу (\ref{eq:D}) и является рекурсивным, а значит, для хранения промежуточных результатов используется стек. Кроме того, при этом подходе возникает проблема повторных вычислений, так как функция $D(s1[1..i], s2[1..j])$ будет выполняться несколько раз в разных ветвях дерева для одних и тех же значений $i$ и $j$.


\subsection{Рекурсивный алгоритм поиска расстояния \\ \mbox{Дамерау~---~Левенштейна с использованием} \\ кеша}

В этом алгоритме используется кеш. Кеш позволяет сохранять результаты промежуточных вычислений и избегать повторных вычислений, что значительно повышает эффективность алгоритма по сравнению с рекурсивным алгоритмом без кеша.
В качестве кеша используется матрица, её ячейки инициализируются значением $-1$. Хотя и этот алгоритм, и итерационный алгоритм используют матрицы, они различаются по способу реализации. В этом алгоритме они работают в противоположном направлении.

\section*{Вывод}
В данном разделе были описаны алгоритмы поиска расстояний Левенштейна и Дамерау~---~Левенштейна. Эти алгоритмы позволяют найти редакционные расстояния для двух строк.

