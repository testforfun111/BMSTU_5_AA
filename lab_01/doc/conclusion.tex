\chapter*{Заключение}
\addcontentsline{toc}{chapter}{Заключение}

Был изучен метод динамического программирования на материале алгоритмов Левенштейна и Дамерау~---~Левенштейна. Также описаны алгоритмы Левенштейна и Дамерау~---~Левенштейна нахождения расстояния между строками, получены практические навыки реализации указанных алгоритмов в матричной и рекурсивных версиях.

Решены все поставленные задачи:

\begin{itemize}
	\item изучены, разработаны и реализованы алгоритмы поиска расстояний Левенштейна и Дамерау~---~Левенштейна;
    \item выполнены замеры алгоритмов процессорного времени работы реализаций алгоритмов;
	\item проведен сравнительный анализ нерекурсивных алгоритмов поиска расстояний Левенштейна и Дамерау~---~Левенштейна;
	\item проведен сравнительный анализ трех алгоритмов поиска расстояний Дамерау~---~Левенштейна.
	\item подготовлен отчет о лабораторной работе.
\end{itemize}

Экспериментально было подтверждено различие во временной эффективности рекурсивной и нерекурсивной реализаций выбранного алгоритма определения расстояния между строками при помощи разработаного программного обеспечения на материале замеров процессорного времени выполнения реализации на варьирующихся длинах строк.
В результате исследований пришёл к выводу, что алгоритм Левенштейна заметно выигрывает Дамерау~---~Левенштейна в 16\% по времени ри длине строк в более 4, рекурсивный алгоритм Дамерау~---~Левенштейна уже при длине строк равной 4 символа проигрывает в 15 раз по времени итерационной и в 10 раз рекурсивной с кешем реализациям, итерационный алгоритм поиска расстояний Дамерау~---~Левенштейна в среднем на 49~\%~--~54~\% быстрее рекурсивного с кешем для длин строк от 0 до 7 символов. Несмотря на то, что итерационные алгоритмы обладают высоким быстродействием, при больших длинах строк они занимают довольно много памяти под матрицу.