\chapter*{Введение}
\addcontentsline{toc}{chapter}{Введение}

Расстояние Левенштейна и расстояние Дамерау~---~Левенштейна, представляют собой метрики, используемые в информатике и обработке текстов для измерения различий между двумя строками или последовательностями символов.

Расстояние Левенштейна, также известное как редакционное расстояние или расстояние редактирования, измеряет минимальное количество операций (вставки, удаления и замены символов), необходимых для преобразования одной строки в другую. Эта метрика была названа в честь советского математика Владимира Левенштейна и широко используется в автоматической обработке текста и компьютерной лингвистике.

Расстояние Дамерау~---~Левенштейна является расширением расстояния Левенштейна, которое также учитывает операцию транспозиции (перестановки соседних символов) как допустимую операцию. Оно получило свое имя в честь американского ученого Фредерика Дамерау, который также внес вклад в развитие этой метрики.

Метод динамического программирования --- это способ решения сложных задач путём разбиения из на более простые подзадачи.

Целью данной лабораторной работы является изучение метода динамического программирования на материале алгоритмов Левенштейна и Дамерау---Левенштейна.
Задачи данной лабораторной работы следующие:
\begin{itemize}
	\item описание расстояний Левенштейна и Дамерау~---~Левенштейна между строками;
	\item разработка алгоритмов нахождения расстояний Левенштейна и Дамерау~---~Левенштейна между строками;
    \item создание программного обеспечения, реализующего перечисленные выше алгоритмы;
    \item проведение сравнительного анализа реализаций алгоритмов по затраченному процессорному времени и памяти;
    \item описание и обоснование полученных результатов в отчете о выполненной лабораторной работе.
\end{itemize}
