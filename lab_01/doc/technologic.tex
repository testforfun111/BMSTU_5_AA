\chapter{Технологическая часть}

В данном разделе будут приведены средства реализации, листинги кода и функциональные тесты.


\section{Средства реализации}
Для реализации алгоритмов был выбран язык программирования \textit{Python}, а в качестве среды разработки - \textit{Visual Studio Code}.
Для замеров процессорного времени использовалась функция \textit{process\_time()} из библиотеки \textit{time}, а для построения графиков - библиотека \textit{matplotlib} \cite{mpl}.

\section{Сведения о модулях программы}
Программа состоит из 4 модулей:
\begin{itemize}
        \item main.py — точка входа;
        \item algorythms.py — хранит реализацию алгоритмов;
        \item compare.py — хранит реализацию системы замера времени;
        \item constant.py — хранит константы.
\end{itemize} 

\section{Реализация алгоритмов}
Ниже представлены реализации алгоритмов поиска расстояний Левенштейна и Дамерау~---~Левенштейна (листинги \ref{lst:lev}--\ref{lst:recwithcache}).

\begin{center}
    \captionsetup{justification=raggedright,singlelinecheck=off}
    \begin{lstlisting}[label=lst:lev,caption=Итерационный алгоритм поиска расстояния Левенштейна]
def DLevenshtein(str1, str2):
	n = len(str1)
	m = len(str2)
	matrix = createMatrix(n + 1, m + 1, 0)
	
	for i in range(0, n + 1):
		for j in range(0, m + 1):
			if (i == 0 or j == 0):
				matrix[i][j] = max(i, j)
			else:
				matrix[i][j] = min(matrix[i - 1][j] + 1,
							matrix[i][j - 1] + 1,
							matrix[i - 1][j - 1] +
							 int(str1[i - 1] != str2[j - 1]))
	
	return matrix
	
\end{lstlisting}
\end{center}


\begin{center}
    \captionsetup{justification=raggedright,singlelinecheck=off}
    \begin{lstlisting}[label=lst:DamLev,caption=Итерационный алгоритм поиска расстояния Дамерау~---~Левенштейна]
def DDamerauLevenshtein(str1, str2):
	n = len(str1)
	m = len(str2)
	matrix = createMatrix(n + 1, m + 1, 0)
	
	for i in range(0, n + 1):
		for j in range(0, m + 1):
			if (i != 0 and j != 0):
				matrix[i][j] = min(matrix[i - 1][j] + 1,
						matrix[i][j - 1] + 1,
						matrix[i - 1][j - 1] + 
							int(str1[i - 1] != str2[j - 1]))
			if (i > 1 and j > 1 and
				str1[i - 1] == str2[j - 2] and
				str1[i - 2] == str2[j - 1]):
					matrix[i][j] = min(matrix[i][j], 
									matrix[i - 2][j - 2] + 1)
			else:
				matrix[i][j] = max(i, j)
		
	return matrix

	
\end{lstlisting}
\end{center}


\begin{center}
    \captionsetup{justification=raggedright,singlelinecheck=off}
    \begin{lstlisting}[label=lst:recDamLev,caption=Рекурсивный алгоритм поиска расстояния Дамерау~---~Левенштейна]
def DDamerauLevenshteinRecursive(str1, str2):
	n = len(str1)
	m = len(str2)
	
	if n != 0:
		if m != 0:
			add = DDamerauLevenshteinRecursive(str1, str2[:m-1]) + 1
			delete = DDamerauLevenshteinRecursive(str1[:n-1], str2) + 1
			change = DDamerauLevenshteinRecursive(str1[:n-1], 
						str2[:m-1]) + int(str1[n- 1] != str2[m - 1])
			
			result = min(add, delete, change)
			
			if (n >= 2 and m >=2 and 
				str1[n-1] == str2[m-2] and
				str1[n-2] == str2[m-1]):
					result = min(result, DDamerauLevenshteinRecursive(str1[:n-2], 
					str2[:m-2]) + 1)
		else:
			return n 
	else:
		return m 
	return result
	
\end{lstlisting}
\end{center}


\begin{center}
    \captionsetup{justification=raggedright,singlelinecheck=off}
    \begin{lstlisting}[label=lst:recwithcache,caption=Рекурсивный алгоритм поиска расстояния Дамерау~---~Левенштейна с кешем]

def Recursive(str1, str2, n, m, matrix):
	if (n != 0 and m != 0):
		if (matrix[n][m] == -1):
			delete = Recursive(str1, str2, n - 1, m, matrix) + 1
			insert = Recursive(str1, str2, n, m - 1, matrix) + 1
			change = Recursive(str1, str2, n - 1,
					 m - 1, matrix) +
					 int(str1[n - 1] != str2[m - 1])
		
		matrix[n][m] = min(insert, delete, change)
		if (n > 1 and m > 1 and str1[n - 1] == str2[m - 2] and
			str1[n - 2] == str2[m - 1]):
				matrix[n][m] = min(matrix[n][m],
					Recursive(str1, str2, n - 2, m - 2, matrix) + 1)
	else:
		matrix[n][m] = max(n, m)
	
	return matrix[n][m]

def DDamerauLevenshteinRecurciveCache(str1, str2):
	n = len(str1)
	m = len(str2)
	
	matrix = createMatrix(n + 1, m + 1, -1)
	
	Recursive(str1, str2, n, m, matrix)
	
	return matrix

\end{lstlisting}
\end{center}


\section{Функциональные тесты}
	В таблице 3.1 приведены функциональные тесты для алгоритмов поиска расстояний Левенштейна и Дамерау~---~Левенштейна. Все тесты пройдены успешно.
\begin{table}[ht]
	\small
	\begin{center}
		\begin{threeparttable}
			\caption{Функциональные тесты}
			\label{tbl:tests}
			\begin{tabular}{|c|c|c|c|c|c|c|}
				\hline
				& \multicolumn{2}{c|}{\bfseries Входные данные}&\multicolumn{4}{c|}{\bfseries Выходные данные}\\\hline
				&&&&\multicolumn{3}{c|}{\bfseries Дамерау — Левенштейн}\\\cline{5-7}
				&\bfseries str\_1&\bfseries str\_2&\bfseries Левенштейн&\bfseries Нерекурсивный&\multicolumn{2}{c|}{\bfseries Рекурсивный}\\\cline{6-7}
				&\bfseries &\bfseries &\bfseries &\bfseries &\bfseries Без кэша&\bfseries С кэшем
				\csvreader{tests.csv}{}
				{\\\hline\csvcoli & \csvcolii & \csvcoliii & \csvcoliv & \csvcolv & \csvcolvi & \csvcolvii} \\
				\hline
			\end{tabular}
		\end{threeparttable}
	\end{center}
\end{table}

\section*{Вывод}

В этом разделе были рассмотрены средства реализации задания, представлены листинги реализации алгоритмов поиска расстояний Левенштейна и Дамерау~---~Левенштейна, а также тесты.
