\chapter{Аналитическая часть}
В этом разделе будет представлено описание сути конвейрной обработки данных и используемых алгоритмов.

\section{Описание конвейерной обработки данных}

Конвейер~\cite{bib1} — способ организации вычислений, используемый в современных процессорах и контроллерах с целью повышения их производительности (увеличения числа инструкций, выполняемых в единицу времени — эксплуатация параллелизма на уровне инструкций), технология, используемая при разработке компьютеров и других цифровых электронных устройств.

Конвейерную обработку можно использовать для совмещения этапов выполнения разных команд.
Производительность при этом возрастает благодаря тому, что одновременно на различных ступенях конвейера выполняются несколько команд. 
Такая обработка данных в общем случае основана на разделении подлежащей исполнению функции на более мелкие части, называемые лентами, и выделении для каждой из них отдельного блока аппаратуры. 
Так, обработку любой машинной команды можно разделить на несколько этапов (лент), организовав передачу данных от одного этапа к следующему.

\section{Описание алгоритмов}

В данной лабораторной работе на основе конвейрной обработки данных будет поиск подстроки в строке методом полного перебора. 
В качестве алгоритмов на каждую из трех лент были выбраны следующие действия.

\begin{itemize}[label=---]
    \item Поиск всех вхождений подстроки в строку.
    \item Поиск всех вхождений развернутой подстроки в строку.
    \item Запись ответов в единый файл.
\end{itemize}

\section*{Вывод}

В этом разделе было рассмотрено понятие конвейрной обработки данных, а также описанно алгоритмы для обработки строки на каждой из трех лент конвейера.