\chapter*{Введение}
\addcontentsline{toc}{chapter}{Введение}

Для увеличения скорости выполнения программ используют параллельные вычисления. Конвейерная обработка данных является популярным приемом при работе с параллельностью. Она позволяет на каждой следующей «линии» конвейера использовать данные, полученные с предыдушего этапа. \cite{bib0}

Конвейер — способ организации вычислений, используемый в современных процессорах и контроллерах с целью повышения их производительности (эксплуатация параллелизма на уровне инструкций). \cite{bib1}

Целью данной лабораторной работы является изучение принципов конвейрной обработки данных.

Для достижения поставленной цели необходимо выполнить следующие задачи:

\begin{itemize}[label=---]
	\item исследовать основы конвейрной обработки данных;
	\item привести схемы алгоритмов, используемых для конвейрной и линейной обработок данных;
	\item определить средства программной реализации;
	\item провести модульное тестирование;
	\item провести сравнительный анализ времени работы алгоритмов;
	\item описать и обосновать полученные результаты в отчете о выполненной лабораторной работе.
\end{itemize}