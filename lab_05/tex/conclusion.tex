\chapter*{Заключение}
\addcontentsline{toc}{chapter}{Заключение}

В результате исследований можно сделать вывод о том, что при большом количестве обрабатываемых строк, а так же при обработке строки большого размера стоит использовать конвейерную реализацию обработки, а не линейную (при обработке  1600 строк с размерами 100 конвейерная быстрее в 3 раза, а при обработке 100 строк с размерами 320 быстрее в 4 раза).

\vspace{5mm}

В ходе выполнения данной лабораторной работы были решены следующие задачи:
\begin{itemize}[label=---]
	\item изучены основы конвейерной обработки данных;
	\item применены изученные основы для реализации конвейерной обработки строк;
	\item определенны средства программной реализаций;
	\item проведены модульное тестирование;
	\item проведены сравнительный анализ линейной и конвейерной реализаций обработки строк; 
	\item описаны и обоснованы полученные результаты в отчете о выполненной лабораторной работе.
\end{itemize}

Поставленная цель была достигнута.