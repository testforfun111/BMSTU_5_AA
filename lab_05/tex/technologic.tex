\chapter{Технологическая часть}

В данном разделе будут приведены средства реализации, реализация алгоритма, а также функциональные тесты.


\section{Средства реализации}

В данной работе для реализации был выбран язык программирования \textit{C++}~\cite{bib2}, так как он предоставляет весь необходимый функционал для выполнения работы. 
Для замера времени работы использовалась функция \textit{std::chrono::system\_clock::now()}~\cite{bib3}.

\section{Реализация алгоритмов}

В листингах \ref{lst:parallel_processing} - \ref{lst:stage_3} представлены функции для конвейерного и ленейного алгоритмов обработки строки.

\begin{center}
\captionsetup{justification=raggedright,singlelinecheck=off}
\begin{lstlisting}[label=lst:parallel_processing,caption=Функция алгоритма конвейерной обработки строки]
void parallel(int size, int cnt, bool is_count)
{
	queue<strings_t> qA1;
	queue<strings_t> qA2;
	queue<strings_t> qA3;
	vector<vector<int>> V1;
	vector<vector<int>> V2;
	bool qA1_is_empty = false;
	bool qA2_is_empty = false;
	mutex m1;
	mutex m2;
	mutex time_mutex;
	
	for (int i = 0; i < cnt; i++)
	{
		strings_t strings;
		strings.text = random_text(size);
		strings.pattern = random_text(size / 4);
		qA1.push(strings);
	}
	
	vector<strings_state_t> state(cnt);
	for (int i = 0; i < cnt; i++)
	{
		strings_state_t tmp_state;
		tmp_state.stage_1 = false;
		tmp_state.stage_2 = false;
		tmp_state.stage_3 = false;
		state[i] = tmp_state;
	}
	chrono::time_point<std::chrono::system_clock> time_begin = chrono::system_clock::now();
	vector<res_time_t> time_result_arr;
	init_time_result_arr(time_result_arr, time_begin, cnt, 3);
	thread threads[3];
	threads[0] = thread(parallel_stage_1, ref(time_mutex), ref(m1), ref(qA1), ref(qA2), ref(V1), ref(state), ref(time_result_arr), ref(qA1_is_empty));
	threads[1] = thread(parallel_stage_2, ref(time_mutex), ref(m1), ref(m2), ref(qA2), ref(qA3), ref(V2), ref(state), ref(time_result_arr), ref(qA1_is_empty), ref(qA2_is_empty));
	threads[2] = thread(parallel_stage_3, ref(time_mutex), ref(m2), ref(qA3), ref(V1), ref(V2), ref(state), ref(time_result_arr), ref(qA2_is_empty));
	
	for (int i = 0; i < 3; i++)
	threads[i].join();
	if (is_count)
	printf("     %4d      |     %4d      |   %.6f  \n",
	size, cnt, time_result_arr[cnt - 1].end);
	else
	print_res_time(time_result_arr, cnt * 3);
}
\end{lstlisting}
\end{center}

\clearpage

\begin{center}
\captionsetup{justification=raggedright,singlelinecheck=off}
\begin{lstlisting}[label=lst:linear_processing,caption=Функция алгоритма линейной обработки строки]
void liner(int size, int cnt, bool is_count)
{
	queue<strings_t> qA1;
	queue<strings_t> qA2;
	queue<strings_t> qA3;
	vector<int> tmpV1;
	vector<int> tmpV2;
	mutex m1;
	mutex m2;
	
	std::chrono::time_point<std::chrono::system_clock> time_start, time_end,
	time_begin = std::chrono::system_clock::now();
	
	std::vector<res_time_t> time_result_arr;
	init_time_result_arr(time_result_arr, time_begin, cnt, 3);
	
	for (int i = 0; i < cnt; i++)
	{
		strings_t strings;
		strings.text = random_text(size);
		strings.pattern = random_text(size / 4);
		
		qA1.push(strings);
	}
	
	for (int i = 0; i < cnt; i++)
	{
		time_start = chrono::system_clock::now();
		tmpV1 = task1(ref(m1), ref(qA1), ref(qA2));
		time_end = std::chrono::system_clock::now();
		
		save_result(time_result_arr, time_start, time_end, time_result_arr[0].time_begin, i + 1, 1);
		
		time_start = chrono::system_clock::now();
		tmpV2 = task2(ref(m1), ref(m2), ref(qA2), ref(qA3));
		time_end = std::chrono::system_clock::now();
		
		save_result(time_result_arr, time_start, time_end, time_result_arr[0].time_begin, i + 1, 2);
		
		time_start = chrono::system_clock::now();
		task3(ref(m2), ref(qA3), ref(tmpV1), ref(tmpV2), Myfile);
		time_end = std::chrono::system_clock::now();
		
		save_result(time_result_arr, time_start, time_end, time_result_arr[0].time_begin, i + 1, 3);
	}
	
	if (is_count)
	{
		printf("     %4d      |     %4d      |   %.6f  \n",
		size, cnt, time_result_arr[cnt - 1].end);
	}
	else
	{
		print_res_time(time_result_arr, cnt * 3);
	}
}
\end{lstlisting}
\end{center}


\begin{center}
\captionsetup{justification=raggedright,singlelinecheck=off}
\begin{lstlisting}[label=lst:parallel_stage_1,caption=Функция 1-ой ленты конвейерной обработки строки]
void parallel_stage_1(mutex& time_mutex, mutex& m1, queue<strings_t>& qA1, queue<strings_t>& qA2, vector<vector<int>>& V1, vector<strings_state_t>& state, vector<res_time_t>& time_result_arr, bool & qA1_is_empty)
{
	chrono::time_point<chrono::system_clock> time_start, time_end;
	int task_num = 1;
	while (!qA1.empty())
	{
		time_start = chrono::system_clock::now();
		vector<int> tmp = task1(m1, qA1, qA2);
		time_end = std::chrono::system_clock::now();
		time_mutex.lock();
		save_result(time_result_arr, time_start, time_end, time_result_arr[0].time_begin, task_num, 1);
		time_mutex.unlock();
		V1.push_back(tmp);
		state[task_num - 1].stage_1 = true;
		task_num++;
	}
	qA1_is_empty = true;
}
\end{lstlisting}
\end{center}

\begin{center}
\captionsetup{justification=raggedright,singlelinecheck=off}
\begin{lstlisting}[label=lst:parallel_stage_2,caption=Функция 2-ой ленты конвейерной обработки строки]
void parallel_stage_2(mutex& time_mutex, mutex& m1, mutex& m2, queue<strings_t>& qA2, queue<strings_t>& qA3, vector<vector<int>>& V2, vector<strings_state_t>& state, vector<res_time_t>& time_result_arr, bool &qA1_is_empty, bool & qA2_is_empty)
{
	chrono::time_point<chrono::system_clock> time_start, time_end;
	int task_num = 1;
	while (1)
	{
		if (qA2.empty() == false)
		{
			if (state[task_num - 1].stage_1 == true)
			{
				time_start = chrono::system_clock::now();
				vector<int> tmp = task2(m1, m2, qA2, qA3);
				time_end = std::chrono::system_clock::now();
				time_mutex.lock();
				save_result(time_result_arr, time_start, time_end, time_result_arr[0].time_begin, task_num, 2);
				time_mutex.unlock();
				V2.push_back(tmp);
				state[task_num - 1].stage_2 = true;
				task_num++;
			}
		}
		else if (qA1_is_empty)
		break;
	}
	qA2_is_empty = true;
}
\end{lstlisting}
\end{center}

\clearpage

\begin{center}
\captionsetup{justification=raggedright,singlelinecheck=off}
\begin{lstlisting}[label=lst:parallel_stage_3,caption=Функция 3-ей ленты конвейерной обработки строки]
void parallel_stage_3(mutex& time_mutex, mutex& m2, queue<strings_t>& qA3, vector<vector<int>>& V1, vector<vector<int>>& V2, vector<strings_state_t>& state, vector<res_time_t>& time_result_arr, bool& qA2_is_empty)
{
	chrono::time_point<chrono::system_clock> time_start, time_end;
	int task_num = 1;
	while (1)
	{
		if (qA3.empty() == false)
		{
			if (state[task_num - 1].stage_2 == true && state[task_num - 1].stage_1 == true)
			{
				time_start = chrono::system_clock::now();
				task3(m2, qA3, V1[task_num - 1], V2[task_num - 1], Myfile);
				time_end = std::chrono::system_clock::now();
				time_mutex.lock();
				save_result(time_result_arr, time_start, time_end, time_result_arr[0].time_begin, task_num, 3);
				time_mutex.unlock();
				state[task_num - 1].stage_3 = true;
				task_num++;
			}
		}
		else if (qA2_is_empty)
		break;
	}
}

\end{lstlisting}
\end{center}

\clearpage

\begin{center}
\captionsetup{justification=raggedright,singlelinecheck=off}
\begin{lstlisting}[label=lst:stage_1,caption=Функция реализации 1-ого этапа обработки строки]
vector<int> task1(mutex& m1, queue<strings_t>& qA1, queue<strings_t>& qA2)
{
	m1.lock();
	strings_t tmpStrings = qA1.front();
	m1.unlock();
	vector<int> tmpV = algoS(tmpStrings.text, tmpStrings.pattern);
	
	m1.lock();
	qA2.push(tmpStrings);
	m1.unlock();
	
	qA1.pop();
	return tmpV;
}
\end{lstlisting}
\end{center}

\begin{center}
\captionsetup{justification=raggedright,singlelinecheck=off}
\begin{lstlisting}[label=lst:stage_2,caption=Функция реализации 2-ого этапа обработки строки]
vector<int> task2(mutex& m1, mutex& m2, queue<strings_t>& qA2, queue<strings_t>& qA3)
{
	
	m1.lock();
	strings_t tmpStrings = qA2.front();
	m1.unlock();
	
	vector<int> tmpV = algoS(tmpStrings.text, reverse_string(tmpStrings.pattern));
	
	m2.lock();
	qA3.push(tmpStrings);
	m2.unlock();
	m1.lock();
	qA2.pop();
	m1.unlock();
	return tmpV;
}
\end{lstlisting}
\end{center}

\clearpage

\begin{center}
\captionsetup{justification=raggedright,singlelinecheck=off}
\begin{lstlisting}[label=lst:stage_3,caption=Функция реализации 3-его этапа обработки строки]
void task3(mutex& m2, queue<strings_t>& qA3, vector<int>& V1, vector<int>& V2, ofstream& file)
{
	m2.lock();
	strings_t tmpStrings = qA3.front();
	m2.unlock();
	
	write_to_file(file, tmpStrings.pattern, tmpStrings.text, V1, V2);
	
	m2.lock();
	qA3.pop();
	m2.unlock();
}
\end{lstlisting}
\end{center}

\section{Функциональные тесты}

В таблице \ref{tbl:functional_test} приведены функциональные тесты для конвейерного и ленейного алгоритмов обработки строки. Все тесты пройдены успешно.

\begin{table}[h]
	\begin{center}
	\begin{threeparttable}
		\captionsetup{justification=raggedright,singlelinecheck=off}
		\caption{\label{tbl:functional_test} Функциональные тесты}
		\begin{tabular}{|c|c|c|c|}
			\hline
			 Размер строки & Кол--во заявок & Алгоритм & Ожидаемый результат 
			\\ \hline
			-10 & 10 & Конвейерный & Сообщение об ошибке 
			\\ \hline
			 10 & -10 & Конвейерный & Сообщение об ошибке 
			\\ \hline
			 0 & 10 & Конвейерный & Сообщение об ошибке 
			\\ \hline
			 k & 10 & Конвейерный & Сообщение об ошибке 
			\\ \hline
			 100 & 20 & Конвейерный & Вывод результ. таблички
			\\ \hline
			 100 & 20 & Линейный & Вывод результ. таблички
			\\ \hline
		\end{tabular}
	\end{threeparttable}
	\end{center}
\end{table}

\section*{Вывод}

В данном разделе были разработаны алгоритмы для конвейерного и ленейного алгоритмов обработки матриц, проведено тестирование, описаны средства реализации.
