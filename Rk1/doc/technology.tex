\chapter{Технологическая часть}

В данном разделе будут указаны средства реализации, будут представлены реализация алгоритма, а также функциональные тесты.



\section{Средства реализации}

Данная программа разработана на языке C++, поддерживаемом многими операционными системами.



\section{Реализация алгоритмов}

Aлгоритм поиска полным перебором ищет соседние 2 элемента приведены в листинге \ref{lst:solution}.


\begin{center}
\captionsetup{justification=raggedright,singlelinecheck=off}
\begin{lstlisting}[label=lst:solution,caption=Aлгоритм поиска полным перебором]
int solution(vector<int> arr, int x1, int x2)
{
	int flag = 0;
	int i = 0;
	for (; i < arr.size() - 1; i++)
	{
		if (arr[i] == x1 &&  arr[i + 1] == x2)
		flag = 1;
		if (flag == 1)
		break;
	}
	if (flag == 1)
	return i;
	else 
	return -1;
}

\end{lstlisting}
\end{center}
\clearpage


\section{Функциональные тесты}

В таблице \ref{tbl:tests} приведены функциональные тесты для функций, реализующих алгоритмы. Все тесты пройдены успешно.

\begin{table}[h]
	\begin{center}
		\begin{threeparttable}
			\captionsetup{justification=raggedright,singlelinecheck=off}
			\caption{\label{tbl:tests} Функциональные тесты}
			\begin{tabular}{|c|c|c|}
				\hline
				№&Входные данные& Ожидаемый результат \\
				\hline
				1&5 1 8 2 3 5 2 3& 2\\
				\hline
				2&5 1 8 2 3 5 2 4& -1\\
				\hline
				3&5 1 8 2 3 5 4 2& Ошибка ввода\\
				\hline
			\end{tabular}
		\end{threeparttable}
	\end{center}
\end{table}


\section{Вывод}

Были реализованы функции поиска полным перебором.
Было проведено функциональное тестирование данного алгоритма.